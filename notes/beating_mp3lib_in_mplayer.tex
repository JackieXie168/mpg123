\documentclass[a4paper,12pt]{scrartcl}
\usepackage{pxfonts}
\usepackage{hyperref}

\begin{document}

\setlength{\parindent}{0pt}
\setlength{\parskip}{0.5em}


\tableofcontents

\section{Introduction}

Over a year ago, I contributed am mpg123 binding for MPlayer to replace the old fork of mpg123 code dubbed "mp3lib". I can say "old" since we incorporated the added optimizations from that one plus added new stuff from Taihei;-)

Generally, mpg123 performs well compared to MPlayer, but there is a performance hit when using libmpg123 from MPlayer, making it slower than the built-in mp3lib. This is mostly apparent on older systems, where possibly cache misses are a source of trouble.

Specifically, the regression has been reported on K6-III+, of which I happen to have a unit available for testing. It has been confirmed for Athlons (no XP), too. Well, I happen to have a good old Duron 800 at my disposal. That should do the trick there.

\section{Test setup}

The software baseline is a current stable debian squeeze (installed from 6.0.3 media), with MPlayer and mpg123 pulled from SVN and built enforcing binaries for a K6-III (also on the Duron), adding in 3DNowExt optimizations via custom assembly on either side.

Furthermore, the data set is, for the sake of comparability with earlier results, the MP3 album "Points of View" by "Convergence". To make sure, I talk of these files with the respective MD5 hashes:

\begin{verbatim}
04d953da593d8cfa43aee4d00572ca37 \
 /dev/shm/convergence_-_points_of_view/01 - Bleed.mp3
2999166bab6fe443f6a2c15c8eb5d938 \
 /dev/shm/convergence_-_points_of_view/02 - Strike the end.mp3
d324e12b3f20bea69c9c5e01252d0d96 \
 /dev/shm/convergence_-_points_of_view/03 - Listen.mp3
e0215065002ef2492a8d88c356609c21 \
 /dev/shm/convergence_-_points_of_view/04 - Six feet under.mp3
2b19190e15b5e5a805ccfacd789b01ab \
 /dev/shm/convergence_-_points_of_view/05 - Always the same.mp3
502c4d1e5793352fb25e6e90d6480415 \
 /dev/shm/convergence_-_points_of_view/06 - Breath.mp3
80bffd4e4bd6cc2f330b08e8156070ad \
 /dev/shm/convergence_-_points_of_view/07 - Vanished memories.mp3
de5ebd6e36eebf8c77642fa52f9610bd \
 /dev/shm/convergence_-_points_of_view/08 - Silent.mp3
c4c1167b5ca29c061d75f6104eec9283 \
 /dev/shm/convergence_-_points_of_view/09 - Nothing else.mp3
df624cee3f7b65be9b98e449e122ac63 \
 /dev/shm/convergence_-_points_of_view/10 - Train to leave.mp3
\end{verbatim}

As you may notice: I try to eliminate the hard disk at least for the test data by moving stuff to RAM. Even the K6-III+ has no issue with sparing some of its 512 MiB of RAM (the duron setup has 768 MiB).

The command lines try to do comparable things in mpg123 and MPlayer:

\begin{verbatim}
mplayer -ac mp3 -quiet -ao pcm:file=/dev/null \
   /dev/shm/convergence_-_points_of_view/*.mp3
mplayer -ac mpg123 -quiet -ao pcm:file=/dev/null \
   /dev/shm/convergence_-_points_of_view/*.mp3
mpg123 -w /dev/null /dev/shm/convergence_-_points_of_view/*.mp3
\end{verbatim}

I'm trying to combine timings with good old 'time' and measurements with 'perf'. The latter does not give much information on the K6-III+ with respect to cache misses, so I hope to do the grunt work on the Duron and hopefully the findings will transfer to the K6.

As an additional measure to avoid a penalty for libmpg123 loading, the libmpg123.a is linked in statically into MPlayer. The final solution will hopefully make the dynamic lib perform about as well (considering that stand-alone mpg123 does this successfully).

The build of MPlayer is configured via
\begin{verbatim}
./configure --extra-cflags='-march=k6-3 -mtune=k6-3 -g'
\end{verbatim}
the mpg123 build via
\begin{verbatim}
CFLAGS="-march=k6-3 -mtune=k6-3 -g -O3" ./configure --disable-shared
\end{verbatim}

That results in this set of CFLAGS: \verb:-O2: \verb:-fomit-frame-pointer: \verb:-funroll-all-loops: \verb:-finline-functions: \verb:-ffast-math:  \verb:-march=k6-3: \verb:-mtune=k6-3: \verb:-g: \verb:-O3:
Those are largely similar to what MPlayer uses, but not identical. Thinking about this... some of those might be even damaging. But: As long as standalone mpg123 has such a clear lead over MPlayer, I doubt that those make much of a difference.


\section{The situation}

How do my K6-III binaries perform on the two systems? Well, let's record the user time to decode the test album (no statistics as long as the result is rather clear).
\begin{center}
\begin{tabular}{lrrr}
run & user time K6-III+ / s & user time Duron / s & ratio K6 / Duron \\
\hline
mpg123-debian & 51.6 & 22.4 & 2.30 \\
mpg123 & 50.4 & 22.3 & 2.26 \\
mplayer-mp3lib & 56.8 & 25.4 & 2.24 \\
mplayer-mpg123 & 67.0 & 28.7 & 2.33
\end{tabular}
\end{center}

This is using run-time choice of 3DNowExt routines. It might be interesting to compare runtimes of generic C code, the difference attesting to the respective assembly routine quality.
But let's stay focused on the case that is important in practice, and that is using 3DNowExt as the most performant choice on these CPUs.\footnote{Alas, that is limited to 16 bit output, currently.}
Let's express runtime as percent of plain mpg123 runtime:
\begin{center}
\begin{tabular}{lrr}
run & user time K6-III+ & user time Duron \\
\hline
mpg123-debian & 102.4 \% & 100.4 \% \\
mpg123 & 100.0 \% & 100.0 \% \\
mplayer-mp3lib & 112.7 \% & 113.9 \% \\
mplayer-mpg123 & 132.9 \% & 128.7 \%
\end{tabular}
\end{center}

Ivan noted a performance penalty of 6 \% on his Athlon. Well, I can offer even more: MPlayer with mpg123 binding adds 13 \% on the runtime of MPlayer with mp3lib. Cache size could explain that somewhat, as the Duron does have less of that than an Athlon, though it was not considered that much of a performance difference back then when people cared about comparing Athlons and Durons.
Anyhow, good starting point to dig at the reason for the difference, eh? I do remember trying out generic decoders on the K6-III+, there the effect was even larger. If things are unclear, I can resort to that.

\section{Glimpses with perf}

Now let's hook up with Linux Performance Events. On the K6-III+, the information was rather meager\footnote{I had to build a kernel for K6 to support performance events at all, as opposed to the debian i486 kernel.}, I hope for more stats on the Duron as a more modern / better supported CPU.
But perhaps it's just my novice usage of perf, which I didn't know before. Oh, hm. Let's see. Ah, yes... all the goodies Ivan used in his perf invocation relies on a PMU being present. That's not the case for the K6.
\begin{verbatim}
altwarentester@k6-3:~/mplayer$ perf record -f \
 -e cycles:u -e instructions:u -e l1-dcache-loads:u \
 -e l1-dcache-load-misses:u -e branch-misses \
 ./mplayer -ac mp3 -quiet -ao pcm:file=/dev/null  \
 /dev/shm/convergence_-_points_of_view/*.mp3

  Error: perfcounter syscall returned with -1 (No such device)

  Fatal: No CONFIG_PERF_EVENTS=y kernel support configured?
\end{verbatim}
It seems the only diagnostic I get is cycles, and that's not very special. On the Duron:
\begin{verbatim}
altwarentester@duron:~/mplayer$ perf record -f \
 -e cycles:u -e instructions:u -e l1-dcache-loads:u \
 -e l1-dcache-load-misses:u -e branch-misses \
 ./mplayer -ac mp3 -quiet -ao pcm:file=/dev/null  \
 /dev/shm/convergence_-_points_of_view/*.mp3
  Error: failed to mmap with 1 (Operation not permitted)
\end{verbatim}
Well. That's nearly success. I have to increase the allowed locked memory count from the default 64 K.

Now let's see what we've got. Hm, recording works, but I don't get how to report all that wonderful information.
Plain \verb:perf report: doesn't tell more than usual. I can do that call graph thing, but also here, there is still some strangeness going on with the mpg123 symbols. I'll figure that out.

In the meantime, there is \verb:perf stat:, which gives a nice overall picture:

\begin{verbatim}
altwarentester@duron:~/mplayer$ perf stat ./mplayer -ac mpg123 \
 -quiet -ao pcm:file=/dev/null    /dev/shm/convergence_-_points_of_view/*.mp3

   29435.786923  task-clock-msecs         #      0.994 CPUs 
            216  context-switches         #      0.000 M/sec
              0  CPU-migrations           #      0.000 M/sec
           6780  page-faults              #      0.000 M/sec
    23502185554  cycles                   #    798.422 M/sec
    22870306002  instructions             #      0.973 IPC  
     8562390418  cache-references         #    290.884 M/sec
       17327364  cache-misses             #      0.589 M/sec

   29.605841362  seconds time elapsed
\end{verbatim}

\begin{verbatim}
altwarentester@duron:~/mplayer$ perf stat ./mplayer -ac mp3 \
 -quiet -ao pcm:file=/dev/null    /dev/shm/convergence_-_points_of_view/*.mp3

   26210.959074  task-clock-msecs         #      0.993 CPUs 
            207  context-switches         #      0.000 M/sec
              0  CPU-migrations           #      0.000 M/sec
           6290  page-faults              #      0.000 M/sec
    20923114721  cycles                   #    798.258 M/sec
    20818859901  instructions             #      0.995 IPC  
     7867755901  cache-references         #    300.170 M/sec
       10962558  cache-misses             #      0.418 M/sec

   26.387378954  seconds time elapsed
\end{verbatim}

\begin{verbatim}
altwarentester@duron:~$ perf stat mpg123/src/mpg123 -w /dev/null \
 /dev/shm/convergence_-_points_of_view/*.mp3

   23163.487659  task-clock-msecs         #      0.996 CPUs 
            140  context-switches         #      0.000 M/sec
              0  CPU-migrations           #      0.000 M/sec
            337  page-faults              #      0.000 M/sec
    18501736597  cycles                   #    798.746 M/sec
    20408582208  instructions             #      1.103 IPC  
     7871105536  cache-references         #    339.807 M/sec
       11304577  cache-misses             #      0.488 M/sec

   23.246529422  seconds time elapsed
\end{verbatim}

The K6 is not that generous with details (not counted from cycles on).
Regarding cache misses, that's 17 million for mplayer-mpg123, 11 million for mplayer-mp3 and a similar 11 million for plain mpg123. OK, it is sub-optimal. I know that. Now how to dig deeper? I do wonder if switching to generic C code would help since this reduces the amount of functions involved.

\section{The situation with generic C code}

Let's try that: Force MPlayer not to activate special optimizations, tell mpg123 to stick to generic decoding. Hm. I see that mp3lib generic code is actually the i386 code. Well, then, two can play that game.

\begin{verbatim}
altwarentester@duron:~$ perf stat mpg123/src/mpg123 --cpu i386 \
 -w /dev/null /dev/shm/convergence_-_points_of_view/*.mp3

   44296.192117  task-clock-msecs         #      0.996 CPUs 
            263  context-switches         #      0.000 M/sec
              0  CPU-migrations           #      0.000 M/sec
            335  page-faults              #      0.000 M/sec
    35386827671  cycles                   #    798.868 M/sec
    34153833893  instructions             #      0.965 IPC  
    10346222151  cache-references         #    233.569 M/sec
       13440804  cache-misses             #      0.303 M/sec

   44.465953954  seconds time elapsed
\end{verbatim}

\begin{verbatim}
altwarentester@duron:~/mplayer-generic$ perf stat ./mplayer -ac mp3 \
  -quiet -ao pcm:file=/dev/null    /dev/shm/convergence_-_points_of_view/*.mp3

   32032.688338  task-clock-msecs         #      0.994 CPUs 
            233  context-switches         #      0.000 M/sec
              0  CPU-migrations           #      0.000 M/sec
           6284  page-faults              #      0.000 M/sec
    25575954000  cycles                   #    798.433 M/sec
    28423479288  instructions             #      1.111 IPC  
     9061572976  cache-references         #    282.885 M/sec
       13821093  cache-misses             #      0.431 M/sec

   32.228199936  seconds time elapsed
\end{verbatim}

\begin{verbatim}
altwarentester@duron:~/mplayer-generic$ perf stat ./mplayer -ac mpg123 \
  -quiet -ao pcm:file=/dev/null    /dev/shm/convergence_-_points_of_view/*.mp3

   49121.685292  task-clock-msecs         #      0.995 CPUs 
            309  context-switches         #      0.000 M/sec
              0  CPU-migrations           #      0.000 M/sec
           6781  page-faults              #      0.000 M/sec
    39219721233  cycles                   #    798.420 M/sec
    35350587000  instructions             #      0.901 IPC  
    10824302596  cache-references         #    220.357 M/sec
       21094060  cache-misses             #      0.429 M/sec

   49.370325203  seconds time elapsed
\end{verbatim}

Wait a moment... This looks a lot like mpg123's C code being mashed into an inferior binary form.
It's still slower from MPlayer than from stand-alone mpg123, but the comparison to mp3lib is harsh.
But then, wait a bit more: Is MPlayer really playing fair? Did I hack mp3lib correctly?
Those 32 seconds aren't that much more than 26 seconds. Let's record.

Hm, well... it looks like generic code, alright. Building mpg123 with \verb:-O4: and the rest of the MPlayer CFLAGS doesn't change too much: runtime stays around 44 seconds for standalone mpg123.
But the comparison is still somewhat interesting: Cache is not the point here.
\begin{verbatim}
altwarentester@duron:~/mpg123-mpopt$ perf stat src/mpg123 --cpu i386 \
 -w /dev/null    /dev/shm/convergence_-_points_of_view/*.mp3

   44294.299679  task-clock-msecs         #      0.996 CPUs 
            272  context-switches         #      0.000 M/sec
              0  CPU-migrations           #      0.000 M/sec
            325  page-faults              #      0.000 M/sec
    35383207571  cycles                   #    798.821 M/sec
    34414257475  instructions             #      0.973 IPC  
    10373238416  cache-references         #    234.189 M/sec
       17922320  cache-misses             #      0.405 M/sec

   44.451762052  seconds time elapsed
\end{verbatim}
Well, that fits with the earlier comparison: Differences in cache hit or miss are not really significant, the mpg123 C code plainly sucks!

So, I got two issues at hand:
\begin{itemize}
	\item With 3DNowExt code, there is a definite penalty that MPlayer gets compared to stand-alone mpg123. This might have to do with CPU cache performance.
	\item The non-assembly mpg123 code doesn't work out well compared to MPlayer's C code. I had a peek and it really looks like the synth function just needs (much) more time.
\end{itemize}
Great. Which one will I fix? I must admit that I am tempted to find out what works differently with the mpg123 C code in its two incarnations. Well, for starters, it could be the change I had to make to give the code thread-safety. That meant movement of quite some data structures from static variables to the handle on the heap.

Let'c check mpg123-0.68, which predates the move to the thread-safe library, with the MPlayer CFLAGS:
\begin{verbatim}
altwarentester@duron:~/mpg123-0.68$ perf stat src/mpg123 --cpu i386 \
 -w /dev/null    /dev/shm/convergence_-_points_of_view/*.mp3

   43475.950745  task-clock-msecs         #      0.996 CPUs 
            264  context-switches         #      0.000 M/sec
              0  CPU-migrations           #      0.000 M/sec
            317  page-faults              #      0.000 M/sec
    34730969332  cycles                   #    798.855 M/sec
    34317362404  instructions             #      0.988 IPC  
    10323913086  cache-references         #    237.463 M/sec
       12847122  cache-misses             #      0.295 M/sec

   43.628917385  seconds time elapsed
\end{verbatim}
Does not look like that much of an extra hit. Well, about 2 \% difference to the current version there. So, it's not nothing.
But in that range I lean to argue that there has to be {\em some} cost involved in making the code thread-safe, with all that data in the handle and al.
Of course, my approach to libmpg123 should have room for cleanup and optimization per se, but we are not talking about 2 \% here, compared to mp3lib. This is more serious than the simple transition to libmpg123.

It is difficult to test earlier versions of mpg123 since they happen not to like my test data. One should not forget that I added bug fixes over time! Well, I suspect my approach to CPU runtime optimization choice.
Found a set of old MP3s to work on (``Dune'', by ... eh... ``Dune'')... and the difference is appaling:

\begin{verbatim}
altwarentester@duron:~/mpg123-0.59r$ perf stat ./mpg123 -t \
 /home/altwarentester/dune/*.mp3

   23312.117777  task-clock-msecs         #      0.996 CPUs 
            152  context-switches         #      0.000 M/sec
              0  CPU-migrations           #      0.000 M/sec
          14344  page-faults              #      0.001 M/sec
    18623038218  cycles                   #    798.857 M/sec
    24254735169  instructions             #      1.302 IPC  
     8597454473  cache-references         #    368.798 M/sec
        4262514  cache-misses             #      0.183 M/sec

   23.394514625  seconds time elapsed
\end{verbatim}
\begin{verbatim}
altwarentester@duron:~/mpg123-0.68$ perf stat ./mpg123 -t \
 /home/altwarentester/dune/*.mp3

   32211.610636  task-clock-msecs         #      0.996 CPUs 
            199  context-switches         #      0.000 M/sec
              0  CPU-migrations           #      0.000 M/sec
            296  page-faults              #      0.000 M/sec
    25731964846  cycles                   #    798.841 M/sec
    30810795898  instructions             #      1.197 IPC  
    10904946603  cache-references         #    338.541 M/sec
        7285056  cache-misses             #      0.226 M/sec

   32.326412050  seconds time elapsed
\end{verbatim}

Frick. That is a huge hit. Hah! I even forgot to specify \verb:--cpu i386:. With that, I'm at around \textbf{70 seconds}.
Something smells fishy.

\section{What happened to mpg123 on the way to 0.68?}

Comparing 0.59r and 0.68 furher... decoding one of the dune tracks, where the old version needs 2.5 seconds and the new one over 6 seconds.
Reports:

\begin{verbatim} 
altwarentester@duron:~/mpg123-0.59r$ perf report
# Samples: 15180
#
# Overhead      Command                  Shared Object  Symbol
# ........  ...........  .............................  ......
#
    45.76%       mpg123  ./mpg123                       [.] III_dequantize_sample
    20.47%       mpg123  ./mpg123                       [.] dct64
    15.01%       mpg123  ./mpg123                       [.] dct36
     9.65%       mpg123  ./mpg123                       [.] do_layer3
     3.41%       mpg123  ./mpg123                       [.] synth_1to1
     1.98%       mpg123  ./mpg123                       [.] dct12
     1.03%       mpg123  /lib/i686/cmov/libc-2.11.2.so  [.] memcpy
\end{verbatim}


\begin{verbatim}
altwarentester@duron:~/mpg123-0.68$ perf report
# Samples: 44859
#
# Overhead      Command                  Shared Object  Symbol
# ........  ...........  .............................  ......
#
    65.42%       mpg123  ./src/mpg123                   [.] synth_1to1_i386
    15.60%       mpg123  ./src/mpg123                   [.] III_dequantize_sample
     7.63%       mpg123  ./src/mpg123                   [.] dct64_i386
     4.91%       mpg123  ./src/mpg123                   [.] dct36
     3.07%       mpg123  ./src/mpg123                   [.] do_layer3
     0.70%       mpg123  ./src/mpg123                   [.] dct12
     0.34%       mpg123  ./src/mpg123                   [.] III_get_scale_factors_1
     0.33%       mpg123  [kernel]                       [k] __copy_to_user_ll
     0.12%       mpg123  [kernel]                       [k] generic_file_aio_read
\end{verbatim}

I seriously borked up the synth function for this CPU. But remember: This is code that is not run in the initially considered case, only when I enforce avoidance of the optimized decoders.
But it could be symptomatic.

\paragraph{Oh, dear.} I listened to some output from 0.59r. It's crap, distorted something.
So, perhaps that just doesn't count. weren't it for the finding that the synth function is more heavy also compared to mp3lib.
Taking a look at mpg123 0.60. Perhaps that's more sane.
No, it isn't. The fun fact is that it works fine on my 64 bit laptop with generic decoder... hm... would generic work?
Yes, the generic decoder works also on the Duron. But it's slow, just about as current mpg123.

Fun question: Does MPLayer with the i386 mp3lib decode correctly, like current mpg123? Yes it does.
Result looks good. Computing time looks better.

Last resort of mpg123 without multi-cpu code: 0.65. Will that produce proper output? Will it be faster? It produces crap. Faster. Only mpg123-0.66 finally produces proper output with i386 decoder. But is that a generic property?
Shouldn't I have noticed such breakage before? I don't see it mentioned in NEWS.

There it is:
\begin{verbatim}
------------------------------------------------------------------------
r577 | thor | 2007-02-22 03:04:23 +0100 (Do, 22. Feb 2007) | 4 Zeilen

fixing i386 code with gcc-4.1 by using the old WRITE_SAMPLE again - the
new one has distorted sound and gcc complains about strict aliasing ...
the old WRITE_SAMPLE even seems to be faster with gcc-3.4!
Of course I didn't (yet) test on my real i386DX40 if there's an advantage.
Started mocking with cpuflags to get proper MMX and SSE detection.

------------------------------------------------------------------------
\end{verbatim}
Well, there it may be. Let's switch around \verb:WRITE_SAMPLE:. And we got a change from
\begin{verbatim}
    2566.163526  task-clock-msecs         #      0.994 CPUs 
             21  context-switches         #      0.000 M/sec
              0  CPU-migrations           #      0.000 M/sec
            300  page-faults              #      0.000 M/sec
     2047880985  cycles                   #    798.032 M/sec
     2041781623  instructions             #      0.997 IPC  
      798943574  cache-references         #    311.338 M/sec
        4049486  cache-misses             #      1.578 M/sec

    2.581971582  seconds time elapsed
\end{verbatim}
to
\begin{verbatim}
    6272.432955  task-clock-msecs         #      0.989 CPUs 
             51  context-switches         #      0.000 M/sec
              0  CPU-migrations           #      0.000 M/sec
            300  page-faults              #      0.000 M/sec
     5006116015  cycles                   #    798.114 M/sec
     4398072978  instructions             #      0.879 IPC  
     1320889896  cache-references         #    210.587 M/sec
        4730705  cache-misses             #      0.754 M/sec

    6.343654290  seconds time elapsed
\end{verbatim}
Very well. Seems like I have to learn from MPlayer about fixing up the new \verb:WRITE_SAMPLE:.
And yes, there is something in revision 7300; something about gcc 3.x being too clever for the double - int conversion trick. Applying the fix from MPlayer to mpg123 0.65 yields
\begin{verbatim}
    5213.636059  task-clock-msecs         #      0.985 CPUs 
             46  context-switches         #      0.000 M/sec
              0  CPU-migrations           #      0.000 M/sec
            301  page-faults              #      0.000 M/sec
     4161133383  cycles                   #    798.125 M/sec
     4203986268  instructions             #      1.010 IPC  
     1302169739  cache-references         #    249.762 M/sec
        4555437  cache-misses             #      0.874 M/sec

    5.293784289  seconds time elapsed
\end{verbatim}
(and good-sounding output, of course).

Going back to the original test album, writing to /dev/null (the stuff before wrote to a real file to check the output):
\begin{verbatim}
   35322.795994  task-clock-msecs         #      0.996 CPUs 
            215  context-switches         #      0.000 M/sec
              0  CPU-migrations           #      0.000 M/sec
            315  page-faults              #      0.000 M/sec
    28215905365  cycles                   #    798.802 M/sec
    32624122756  instructions             #      1.156 IPC  
    10324164674  cache-references         #    292.281 M/sec
       12229060  cache-misses             #      0.346 M/sec

   35.457957266  seconds time elapsed
\end{verbatim}
That's not 70 seconds. \verb:WRITE_SAMPLE: is a sore spot, but that it is that influential on the Duron is a surprise. I have to check on a more modern system, with a 32 bit install.
But where came the 70 seconds from anyway? The current mpg123 had only about 44 seconds. This CPU is sensitive. And don't forget: I observed the old \verb:WRITE_SAMPLE: even being faster back then!

Anyhow, 35 seconds is still above the 32 seconds mp3lib achieves. There is more to the generic/i386 C code. Stay tuned.

Found MPlayer rev. 28866. Reducing the size of bandInfoStruct bring that:
\begin{verbatim}
   35303.691456  task-clock-msecs         #      0.996 CPUs 
            215  context-switches         #      0.000 M/sec
              0  CPU-migrations           #      0.000 M/sec
            314  page-faults              #      0.000 M/sec
    28199408110  cycles                   #    798.767 M/sec
    32624433817  instructions             #      1.157 IPC  
    10330702614  cache-references         #    292.624 M/sec
       14060449  cache-misses             #      0.398 M/sec

   35.430962884  seconds time elapsed
\end{verbatim}
In short: nothing. The preceeding rev. 28865 has some \verb:const: stuff. That might play a role. Then there is Zuxy's rev. 23485 that changed some static vars to automatic. Generally, he worked on SSE stuff, that's not of interest right now.
There is the \verb:fast_memcpy: business. I should check that. Rev. 22158 does buffer alignment.
Revision 18946 fixes something in the 3DNow DCT64 ... and makes we wonder what exactly is the relation of mpg123's and MPlayers plain 3DNow code. That is another matter to investigate.

For now, it seems that it is fruitful enough to pursue merging of apparent improvements in the generic/overall C code of mp3lib that we indeed still miss in mpg123.

Hacking the ``new'' \verb:WRITE_SAMPLE: into current mpg123 does yield a tiny improvement over the hacked 0.65, even.
\begin{verbatim}
altwarentester@duron:~/mpg123$ perf stat ./src/mpg123 -w /dev/null \
 --cpu i386  /dev/shm/convergence_-_points_of_view/*.mp3

   35030.997934  task-clock-msecs         #      0.996 CPUs
            225  context-switches         #      0.000 M/sec
              0  CPU-migrations           #      0.000 M/sec
            333  page-faults              #      0.000 M/sec
    27983581759  cycles                   #    798.823 M/sec
    33233941274  instructions             #      1.188 IPC
    10325844134  cache-references         #    294.763 M/sec
       11774779  cache-misses             #      0.336 M/sec
   
   35.176724921  seconds time elapsed
\end{verbatim}
\begin{verbatim}
altwarentester@duron:~/mpg123-0.65$ perf stat ./src/mpg123 -w /dev/null \
 /dev/shm/convergence_-_points_of_view/*.mp3

   35215.369977  task-clock-msecs         #      0.995 CPUs
            226  context-switches         #      0.000 M/sec
              0  CPU-migrations           #      0.000 M/sec
            314  page-faults              #      0.000 M/sec
    28126526481  cycles                   #    798.700 M/sec
    32624128394  instructions             #      1.160 IPC
    10335841084  cache-references         #    293.504 M/sec
       12821386  cache-misses             #      0.364 M/sec
   
   35.387834336  seconds time elapsed
\end{verbatim}
Too bad that this doesn't fit the current framework of separated clipping and conversion too well. Seems like, at least for some CPUs (I really should wire up the i486, or even the i386...), this separation hurts.
But then: We are talking about the i386 C code, which is not used by default from old Pentiums onwards. This is rather academic. The performance of the plain C code doesn't matter at all in practice, when there's nowadays a variant of Taihei's SSE code in use.

In MPlayer, this change improves the situation, but it is still 39 seconds against 32 seconds:
\begin{verbatim}
perf stat ./mplayer -ac mp3 -ao pcm:file=/dev/null -quiet \
 /dev/shm/convergence_-_points_of_view/*.mp3

   32049.354400  task-clock-msecs         #      0.994 CPUs 
            249  context-switches         #      0.000 M/sec
              0  CPU-migrations           #      0.000 M/sec
           6293  page-faults              #      0.000 M/sec
    25589163911  cycles                   #    798.430 M/sec
    28424067520  instructions             #      1.111 IPC  
     9060506927  cache-references         #    282.705 M/sec
       12323035  cache-misses             #      0.385 M/sec

   32.251952962  seconds time elapsed
\end{verbatim}
\begin{verbatim}
perf stat ./mplayer -ac mpg123 -ao pcm:file=/dev/null -quiet \
 /dev/shm/convergence_-_points_of_view/*.mp3

   39542.041678  task-clock-msecs         #      0.994 CPUs 
            298  context-switches         #      0.000 M/sec
              0  CPU-migrations           #      0.000 M/sec
           6790  page-faults              #      0.000 M/sec
    31575386518  cycles                   #    798.527 M/sec
    34341985738  instructions             #      1.088 IPC  
    10817342380  cache-references         #    273.566 M/sec
       17943896  cache-misses             #      0.454 M/sec

   39.787800467  seconds time elapsed
\end{verbatim}

\section{Why does GCC hate my code?}

But how does the picture of CPU usage per function look now?
With mp3lib:
\begin{verbatim}
    36.40%      mplayer  ./mplayer                            [.] synth_1to1
    28.16%      mplayer  ./mplayer                            [.] III_dequantize_sample
    10.59%      mplayer  ./mplayer                            [.] dct64_1
     9.05%      mplayer  ./mplayer                            [.] dct36
     4.23%      mplayer  ./mplayer                            [.] T.293
     3.92%      mplayer  /lib/i686/cmov/libc-2.11.2.so        [.] memcpy
\end{verbatim}
With libmpg123:
\begin{verbatim}
    40.43%      mplayer  ./mplayer                            [.] INT123_synth_1to1_i386
    22.31%      mplayer  ./mplayer                            [.] III_dequantize_sample
     7.68%      mplayer  ./mplayer                            [.] INT123_dct36
     6.73%      mplayer  ./mplayer                            [.] INT123_dct64_i386
     6.07%      mplayer  /lib/i686/cmov/libc-2.11.2.so        [.] memcpy
     3.51%      mplayer  ./mplayer                            [.] INT123_synth_ntom
     3.35%      mplayer  ./mplayer                            [.] INT123_do_layer3
     1.25%      mplayer  ./mplayer                            [.] feed_read
\end{verbatim}
That's still quite some overhang for the synth function itself.
And even if it's totally irrelevant in practice... I want to see how I messed up the synth function for Durons.

\paragraph{A quick look on the Core2Duo in 32 bit chroot:}
An mpg123 binary built with '-march=pentium3' and mpg123 default optimization shows perhaps {\em some} difference regarding the \verb:WRITE_SAMPLE: change. using i386 decoder.
I might be convinced to give it 2 \% advantage.
It's harder to tell a difference when using the generic decoder, but there might be one. When building for \verb:-march=pentium:, both differences are around 1 to 1.5 \%.
One would need some statistics for more definite answer, but the main point is: The effect, if any, is much less pronounced on the Core2Duo machine. Might be a reason for me not spotting anything wrong. The message behind this, of course, is that tuning things for the Duron should not hurt performance on newer machines (which use SSE optimization anyway, d'oh).

\paragraph{A root cause}
of the slowness seems to be some inefficient translation of \verb:WRITE_SAMPLE: resulting in floating point stack messiness.
Seems like the summand for the \verb:fadd: is kept on the fp stack in mp3lib, but is grabbed from main stack repeatedly for mpg123. One indicator is the appearance of the \verb:fxch: instruction in mpg123's \verb:synth_1to1: function.
It is totally absent in mp3lib's version.

Now the question is how I can rearrange to code to avoid this. I am working on the 0.65 source to stay closer to mp3lib.

\paragraph{A cause to the cause?}
Yes, there might be a simple explanation: The GCC optimizer is pushed along different ways because the mpg123 code is built to separate object files, with public functions, while mp3lib pushes everything into a single object, with stuff like \verb:synth_1to1: being all-static.
That eases optimzation of (floating point) stack usage.

An interesting side note is that for the mpg123 code, the floating point stack messup seems to be pushed by enabling optimization at all.

But anyway, repackaging the stuff to have the direct calls to the synth functions together with the (then static) functions might do the trick.
For modern mpg123 this would mean packaging all synth functions and \verb:layer[123].c: files together with \verb:optimise.c:, perhaps \verb:frame.c:, into one compilation unit.
An indicator could be this: If there are public symbols that are not part of the API in a static libmpg123, those symbols should have been static and {\em not} public.

Of course, performance is only one possible reason to make these functions static:
It is bad library design to have those symbols floating around and possibly colliding when someone uses the static library.

I tried this on the Core2Duo and, if anything, this makes things perhaps slightly worse. On the Duron the tendency to worsen the performance is also indicated. At least the more static build is about 0.5 K smaller. The effect of adapting Dan's patch for const huffman tables is bigger (both in performance and binary size).
But still, this is not really big. I need to get to the core of the bad WRITE\_SAMPLE translation. How does the compiler get the hint from MPlayer code? And... how about building with the Intel or open64 compilers for a change?
Well, one bummer is that I have those installed for 64 bits, not 32.

On the Duron, the effect of the const tables is also quite small.

\section{Revisit after some time...}

Just make sure I still know what I am talking about, mpg123-0.65-like-mplayer r3005, my old MPlayer checkout, both with i386 C code. Perf says for MPlayer:
\begin{verbatim}
   32331.247743  task-clock-msecs         #      0.994 CPUs 
            371  context-switches         #      0.000 M/sec
              0  CPU-migrations           #      0.000 M/sec
           6301  page-faults              #      0.000 M/sec
    25814521225  cycles                   #    798.439 M/sec
    28424759003  instructions             #      1.101 IPC  
     9068719643  cache-references         #    280.494 M/sec
       13063911  cache-misses             #      0.404 M/sec

   32.523356972  seconds time elapsed

# Samples: 270005
#
# Overhead      Command                        Shared Object  Symbol
# ........  ...........  ...................................  ......
#
    35.82%      mplayer  ./mplayer                            [.] synth_1to1
    28.14%      mplayer  ./mplayer                            [.] III_dequantize_sample
    10.57%      mplayer  ./mplayer                            [.] dct64_1
     9.18%      mplayer  ./mplayer                            [.] dct36
     4.19%      mplayer  ./mplayer                            [.] T.293
     4.07%      mplayer  /lib/i686/cmov/libc-2.11.2.so        [.] memcpy
     0.56%      mplayer  ./mplayer                            [.] fast_memcpy
\end{verbatim}
And for mpg123:
\begin{verbatim}
   34845.338614  task-clock-msecs         #      0.996 CPUs
            377  context-switches         #      0.000 M/sec
              0  CPU-migrations           #      0.000 M/sec
            245  page-faults              #      0.000 M/sec
    27835529165  cycles                   #    798.831 M/sec
    32659069064  instructions             #      1.173 IPC
    10169107285  cache-references         #    291.836 M/sec
        8112060  cache-misses             #      0.233 M/sec
   
   34.982723275  seconds time elapsed

# Samples: 292093
#
# Overhead      Command                        Shared Object  Symbol
# ........  ...........  ...................................  ......
#
    51.58%       mpg123  ./src/mpg123                         [.] synth_1to1
    24.78%       mpg123  ./src/mpg123                         [.] III_dequantize_sample
     8.20%       mpg123  ./src/mpg123                         [.] dct64
     7.84%       mpg123  ./src/mpg123                         [.] dct36
     3.73%       mpg123  ./src/mpg123                         [.] do_layer3
     0.52%       mpg123  [kernel]                             [k] __copy_to_user_ll
     0.31%       mpg123  ./src/mpg123                         [.] dct12
     0.25%       mpg123  ./src/mpg123                         [.] III_get_scale_factors_1
     0.17%       mpg123  /lib/i686/cmov/libc-2.11.2.so        [.] __mempcpy
\end{verbatim}
It is still apparent that synth of mpg123 needs more time. 51.6\% of 35 seconds, that is, 18 seconds.
For MPlayer, that's 35.8\% of 32.5 secods. 11.6 seconds.
Let's summarize that.
\begin{center}
\begin{tabular}{lrr}
function & time MPlayer / s & time mpg123 / s
\\\hline
synth\_1to1 & 11.64 & 18.04 \\
III\_dequantize\_sample & 9.15 & 8.67 \\
dct64 & 3.44 & 2.87 \\
dct36 & 2.99 & 2.74 \\
T.293 / do\_layer3 & 1.36 & 1.30
\end{tabular}
\end{center}

That does look interesting. If only the synth function were faster, mpg123 would easily beat mp3lib. All other main functions are a tad quicker, even, than the mp3lib counterparts!
The \verb:perf annotate: ist not too helpful in placing the blame on certain code portions, but I know already what the main difference is. Or do I?
The annotations show huge runtime hit for the faddp instruction in mpg123, where it doesn't really count in MPlayer, at the sample places.

The output of \verb:perf annotate: is confusing. The association of code and assembly seems rather whacky. But first, what is {\em that}?
\begin{verbatim}
     :          for (j=16;j;j--,b0+=0x10,window+=0x20,samples+=step)
0.14 :       806ca36:       83 c3 40                add    $0x40,%ebx
0.64 :       806ca39:       83 c2 40                add    $0x40,%edx
0.01 :       806ca3c:       83 e9 80                sub    $0xffffff80,%ecx
0.22 :       806ca3f:       83 c6 04                add    $0x4,%esi
1.11 :       806ca42:       81 fb 00 04 00 00       cmp    $0x400,%ebx
0.00 :       806ca48:       0f 84 22 01 00 00       je     806cb70 <synth_1t
0.10 :       806ca4e:       dd d8                   fstp   %st(0)
0.83 :       806ca50:       dd d8                   fstp   %st(0)
0.12 :       806ca52:       dd d8                   fstp   %st(0)
0.02 :       806ca54:       dd d8                   fstp   %st(0)
0.47 :       806ca56:       dd d8                   fstp   %st(0)
0.09 :       806ca58:       dd d8                   fstp   %st(0)
\end{verbatim}
Such floating point stack messing around is not to be found, in MPlayer. Then, the detail:
\begin{verbatim}
     :            sum += window[0x4] * b0[0x4];
1.18 :       82478f2:       d9 42 20                flds   0x20(%edx)
0.60 :       82478f5:       d8 49 20                fmuls  0x20(%ecx)
0.11 :       82478f8:       de c1                   faddp  %st,%st(1)
     :            sum -= window[0x5] * b0[0x5];
0.71 :       82478fa:       d9 42 28                flds   0x28(%edx)
0.46 :       82478fd:       d8 49 28                fmuls  0x28(%ecx)
0.17 :       8247900:       de c1                   faddp  %st,%st(1)
\end{verbatim}
And a similar portion from mpg123:
\begin{verbatim}
         :            sum += window[0x4] * b0[0x4];
    0.36 :       806caca:       d9 44 24 34             flds   0x34(%esp)
    0.00 :       806cace:       d8 49 20                fmuls  0x20(%ecx)
    1.90 :       806cad1:       de c7                   faddp  %st,%st(7)
         :            sum -= window[0x5] * b0[0x5];
    0.35 :       806cad3:       d9 44 24 2c             flds   0x2c(%esp)
    0.00 :       806cad7:       d8 49 28                fmuls  0x28(%ecx)
    2.09 :       806cada:       de c7                   faddp  %st,%st(7)
\end{verbatim}
The exact attribution to one instruction might be tough here, but it is apparent that very similar pieces of code perform consistently different. The location on the floating point stack is different.
And, hm, why are array offsets different?
This pattern is carried through: The MPlayer code has matching offsets on the load and multiplication operations, especially matching the array offsets in the C code.
In mpg123, that is different.


\paragraph{Comparing the code for the two synth functions,}
I see only one difference at all: The possible call of the equalizing function.
Hm, it might be worth a try to remove that bit of code. It introduces a branch... and inserts code for a possible function call.
But no, that doesn't make a difference. I'm relieved.

Now, that leaves the strange decisions of the compiler on the table. It does {\em weird} stuff with the floating point sums. Generally, the floating point code is messy. What's the magic compiler switch to change that?

I added some disassembly and the corresponding source to SVN, the public may have a peek at \url{http://mpg123.orgis.org/cgi-bin/scm/mpg123/notes/beating_mp3lib_in_mplayer_material/}.

Separating out the synth function in the MPlayer code into its own object file (not static, obviously) yields binary code very similar to what gcc produces for mpg123.

\paragraph{A proposal for different clipping}
from Ivan didn't do much good on the Duron, actually worsened it a bit. The idea was to save one if branch by  addition and bitmask comparison, but this is not guaranteed to be faster...
Anyhow, this is tangential to the question why the MPlayer build works better.

\paragraph{Back to the assembly line},
my gaze wanders towards the differences outlined above again, and I realize that the array index strangeness might have to do with the fact that there is \%esp involved in mpg123. 
The function stack versus a pointer in a separate register.
Oh, also, I see a trigger for the bad code: Removing the call to DCT64 improves things, not entirely surprisingly.
This is a piece of diff between the separated-out synth function without and with call to non-static dct64:
\begin{verbatim}
119,121c170,172
<   93: d9 41 20                flds   0x20(%ecx)
<   96: d8 4a 20                fmuls  0x20(%edx)
<   99: de c2                   faddp  %st,%st(2)
---
>  109: d9 44 24 28             flds   0x28(%esp)
>  10d: d8 4a 20                fmuls  0x20(%edx)
>  110: de c7                   faddp  %st,%st(7)
123,125c174,176
<   9b: d9 41 28                flds   0x28(%ecx)
<   9e: d8 4a 28                fmuls  0x28(%edx)
<   a1: de c2                   faddp  %st,%st(2)
---
>  112: d9 44 24 20             flds   0x20(%esp)
>  116: d8 4a 28                fmuls  0x28(%edx)
>  119: de c7                   faddp  %st,%st(7)
\end{verbatim}
Now, as I said, a function call messing with stack/register handling is not entirely spurprising.
But, still, I should wrap my head around what exactly happens here.
Especially, since including static dct64 didn't seem to help mpg123. But what does it do here? Let's try.

Gah! Static! I forgot that there are static variables in the synth function (there aren't anymore for current mpg123). I simply defined the static keyword to nothing for the separated object, only thinking about the static function, not static variables.
Does the situation change when I just remove the static keyword from the function?
\begin{verbatim}
156,158c223,225
<   b2: d9 42 20                flds   0x20(%edx)
<   b5: d8 49 20                fmuls  0x20(%ecx)
<   b8: de c1                   faddp  %st,%st(1)
---
>  144: d9 44 24 34             flds   0x34(%esp)
>  148: d8 49 20                fmuls  0x20(%ecx)
>  14b: de c7                   faddp  %st,%st(7)
160,162c227,229
<   ba: d9 42 28                flds   0x28(%edx)
<   bd: d8 49 28                fmuls  0x28(%ecx)
<   c0: de c1                   faddp  %st,%st(1)
---
>  14d: d9 44 24 2c             flds   0x2c(%esp)
>  151: d8 49 28                fmuls  0x28(%ecx)
>  154: de c7                   faddp  %st,%st(7)
\end{verbatim}
OK, that's still the same, basically. Only the stack location changes.
Now, how does the stuff land on the stack anyways?

\paragraph{An indication of victory}
is that, after simply disabling the call to dct64 (and, thusly, producing rubbish) both in mp3lib and mpg123, the mpg123 binary beats MPlayer with 25 agains 28 seconds decoding time.
This function call messes up the code optimization for mpg123 and somehow does not for mp3lib.

This is creepy. I hacked around with the mp3lib code for a while, making functions non-static, cleaning away distracting parts... and then this:
The optional optimized synth function is disabled at runtime by my hacking, but the call was still present in the synth.
\begin{verbatim}
  if ( synth_func )
   {
//    printf("Calling %p, bandPtr=%p channel=%d samples=%p\n",synth_func,bandPtr,channel,samples);
    // FIXME: synth_func() may destroy EBP, don't rely on stack contents!!!
    return (*synth_func)( bandPtr,channel,samples);
   }
\end{verbatim}
Now, if I remove that function call,
\begin{verbatim}
  if ( synth_func )
   {
//    printf("Calling %p, bandPtr=%p channel=%d samples=%p\n",synth_func,bandPtr,channel,samples);
    // FIXME: synth_func() may destroy EBP, don't rely on stack contents!!!
    /* return (*synth_func)( bandPtr,channel,samples); */
   }
\end{verbatim}
I suddenly get my bad code delivered! Let's check the actual performance... Yup, with the call present (but inactive), we have the usual 32 seconds, with the call removed, it's 40 seconds.

OK... so I see now how I could fix the generic decoding to be better. Via some nasty trick I might convince gcc to produce the correct output for mpg123; or, I even capture the good assembly and have the i386 code as assembly optimization, which is a bit crazy, given that there is the i586 optimization anyway.

But first, let's verify the concept, I added this little fake function someplace else than decode\_i386.c (so that the symbol exists and the function call code actually is generated for lack of link-time inlining).
\begin{verbatim}
int synth_1to1_fake(real *bandPtr,int channel,unsigned char *out,int *pnt)
{
	return 0;
}
\end{verbatim}
Then, the bogus fake call gets added to the synth.
\begin{verbatim}
int moody_switch = 0;
DECODE_SCOPE int synth_1to1(real *bandPtr,int channel,unsigned char *out,int *pnt)
{
#ifndef PENTIUM_OPT
  static real buffs[2][2][0x110];
  static const int step = 2;
  static int bo = 1;
  short *samples = (short *) (out + *pnt);

  real *b0,(*buf)[0x110];
  int clip = 0; 
  int bo1;

  *pnt += 128; /* moved that here as MPlayer has it */
#endif

  if(moody_switch)
    return synth_1to1_fake(bandPtr, channel, out, pnt);
\end{verbatim}
Well, the result...
\begin{verbatim}
   28522.600345  task-clock-msecs         #      0.996 CPUs 
            302  context-switches         #      0.000 M/sec
              0  CPU-migrations           #      0.000 M/sec
            246  page-faults              #      0.000 M/sec
    22784727766  cycles                   #    798.831 M/sec
    27729388895  instructions             #      1.217 IPC  
     9218523630  cache-references         #    323.201 M/sec
        7898355  cache-misses             #      0.277 M/sec

   28.650573967  seconds time elapsed
\end{verbatim}

Nifty, eh? Now I have a grip on how to hack utmost speed out of the generic C code on x86 with gcc. That is an interesting point, but it is not the main matter of this investigation.
Let's look at what the binding of libmpg123 to mplayer does to hurt performance by 30\%.

Pack your things and go, mp3lib, you're owned! Well, owned by plain mpg123, that is. Libmpg123 still has some way to go.

\section{Back to the Binding}

Without the hands tied behind its back, mpg123 is faster than MPlayer (using 3DNowExt optimizations).
But the binding of libmpg123 to MPlayer brings great suffering.
Let's investigate that.

MPlayer's mp3lib has such a profile,
\begin{verbatim}
# Samples: 217361
#
# Overhead      Command                        Shared Object  Symbol
# ........  ...........  ...................................  ......
#
    34.08%      mplayer  ./mplayer                            [.] III_dequantize_sample
    17.89%      mplayer  ./mplayer                            [.] synth_1to1_MMX
    16.40%      mplayer  ./mplayer                            [.] dct36_3dnowex
    11.61%      mplayer  ./mplayer                            [.] dct64_MMX_3dnowex
     5.27%      mplayer  ./mplayer                            [.] T.293
     4.85%      mplayer  /lib/i686/cmov/libc-2.11.2.so        [.] memcpy
\end{verbatim}
while with libmpg123 usage, it is
\begin{verbatim}
# Samples: 248096
#
# Overhead      Command                        Shared Object  Symbol
# ........  ...........  ...................................  ......
#
    30.14%      mplayer  ./mplayer                            [.] III_dequantize_sample
    26.41%      mplayer  ./mplayer                            [.] 0x000000009470c1
     9.41%      mplayer  ./mplayer                            [.] 0x00000000950941
     8.04%      mplayer  /lib/i686/cmov/libc-2.11.2.so        [.] memcpy
     7.30%      mplayer  ./mplayer                            [.] INT123_synth_ntom
     4.62%      mplayer  ./mplayer                            [.] INT123_do_layer3
     1.69%      mplayer  ./mplayer                            [.] feed_read
     1.15%      mplayer  ./mplayer                            [.] INT123_dct64
     0.97%      mplayer  ./mplayer                            [.] INT123_synth_1to1_3dnowext
\end{verbatim}
Wait. What?! Ntom? And the actual synth with 0.97\%? That is fishy.
The figures may be right, but the names aren't. Perhaps the static link messes things up a bit (but why should it?).
Let's link to the system libmpg123.
\begin{verbatim}
# Samples: 225783
#
# Overhead      Command                        Shared Object  Symbol
# ........  ...........  ...................................  ......
#
    82.32%      mplayer  /usr/lib/libmpg123.so.0.25.1         [.] 0x00000000028247
     7.92%      mplayer  /lib/i686/cmov/libc-2.11.2.so        [.] memcpy
     1.00%      mplayer  /usr/lib/libmpg123.so.0.25.1         [.] 0x000000000147c4
     0.65%      mplayer  [kernel]                             [k] __copy_to_user_ll
     0.53%      mplayer  /lib/i686/cmov/libc-2.11.2.so        [.] 0x000000000705dd
     0.31%      mplayer  ./mplayer                            [.] avpriv_ac3_parse_header
     0.31%      mplayer  /lib/i686/cmov/libc-2.11.2.so        [.] __mempcpy
     0.28%      mplayer  ./mplayer                            [.] ac3_eac3_probe
     0.23%      mplayer  ./mplayer                            [.] demux_audio_fill_buffer
     0.19%      mplayer  /usr/lib/libmpg123.so.0.25.1         [.] mpg123_decode
\end{verbatim}
Whut? Get out already. What is that?
Ok, fresh build, dynamic link to local libmpg123.
\begin{verbatim}
# Samples: 245675
#
# Overhead      Command                 Shared Object  Symbol
# ........  ...........  ............................  ......
#
    36.09%      mplayer  .libs/libmpg123.so.0.33.1     [.] 0x00000000043685
    31.83%      mplayer  .libs/libmpg123.so.0.33.1     [.] III_dequantize_sample
     8.23%      mplayer  /lib/i686/cmov/libc-2.11.2.so [.] memcpy
     4.86%      mplayer  .libs/libmpg123.so.0.33.1     [.] INT123_synth_ntom
     4.65%      mplayer  .libs/libmpg123.so.0.33.1     [.] INT123_do_layer3
     1.67%      mplayer  .libs/libmpg123.so.0.33.1     [.] feed_read
     1.25%      mplayer  .libs/libmpg123.so.0.33.1     [.] INT123_dct64
     1.06%      mplayer  .libs/libmpg123.so.0.33.1     [.] INT123_synth_1to1_3dnowext
\end{verbatim}
Wtf? First: This needs more time than system mpg123, second: Those symbols can't be right.

Checking this... Heck, the NtoM synth {\em is} called! And, know what? It's because of a bug in the libmpg123 parser in feeder mode (I really should not have added this contraption):
It did not seek back 4 bytes if the ahead check past the first frame does not work.
This pecuilar bug is triggered because MPlayer also parses the MPEG data and manages to give exactly one frame of data at the beginning.

Now, for some more sensible numbers and names...
\begin{verbatim}
   27269.086270  task-clock-msecs         #      0.993 CPUs 
            322  context-switches         #      0.000 M/sec
              0  CPU-migrations           #      0.000 M/sec
           6858  page-faults              #      0.000 M/sec
    21765861127  cycles                   #    798.188 M/sec
    20944822745  instructions             #      0.962 IPC  
     8198643751  cache-references         #    300.657 M/sec
       16856742  cache-misses             #      0.618 M/sec

   27.465041648  seconds time elapsed

# Samples: 228283
#
# Overhead      Command                  Shared Object  Symbol
# ........  ...........  .............................  ......
#
    34.38%      mplayer  .libs/libmpg123.so.0.33.1      [.] III_dequantize_sample
    30.03%      mplayer  .libs/libmpg123.so.0.33.1      [.] 0x00000000043e7d
    10.94%      mplayer  .libs/libmpg123.so.0.33.1      [.] 0x00000000042e2d
     7.84%      mplayer  /lib/i686/cmov/libc-2.11.2.so  [.] memcpy
     4.97%      mplayer  .libs/libmpg123.so.0.33.1      [.] INT123_do_layer3
     1.23%      mplayer  .libs/libmpg123.so.0.33.1      [.] INT123_synth_1to1_3dnowext
     0.66%      mplayer  .libs/libmpg123.so.0.33.1      [.] III_get_scale_factors_1
     0.63%      mplayer  [kernel]                       [k] __copy_to_user_ll
\end{verbatim}
Now that looks better, but only covers a bug I must have introduced recently (yes, revision 2935 must've been it). Those 27.5 seconds are still rather bad compared to plain mpg123:
\begin{verbatim}
   24365.715653  task-clock-msecs         #      0.991 CPUs 
            372  context-switches         #      0.000 M/sec
              0  CPU-migrations           #      0.000 M/sec
           7600  page-faults              #      0.000 M/sec
    19462031681  cycles                   #    798.747 M/sec
    20804412868  instructions             #      1.069 IPC  
     8193661424  cache-references         #    336.278 M/sec
       13213580  cache-misses             #      0.542 M/sec

   24.576109613  seconds time elapsed

# Samples: 200099
#
# Overhead      Command                  Shared Object  Symbol
# ........  ...........  .............................  ......
#
    46.39%    lt-mpg123  .libs/libmpg123.so.0.33.1      [.] 0x000000000413ac
    38.52%    lt-mpg123  .libs/libmpg123.so.0.33.1      [.] III_dequantize_sample
     5.81%    lt-mpg123  .libs/libmpg123.so.0.33.1      [.] INT123_do_layer3
     1.45%    lt-mpg123  .libs/libmpg123.so.0.33.1      [.] INT123_synth_1to1_3dnowext
     0.78%    lt-mpg123  [kernel]                       [k] __copy_to_user_ll
     0.74%    lt-mpg123  /lib/i686/cmov/libc-2.11.2.so  [.] __mempcpy
     0.66%    lt-mpg123  .libs/libmpg123.so.0.33.1      [.] III_get_scale_factors_1
\end{verbatim}
I must assume that this first block encompasses all synth functions combined (asm aynth and the DCTs).

How is the absolute comparison to mp3lib, per function? First the raw data again...
\begin{verbatim}
   26190.932710  task-clock-msecs         #      0.993 CPUs 
            333  context-switches         #      0.000 M/sec
              0  CPU-migrations           #      0.000 M/sec
           6307  page-faults              #      0.000 M/sec
    20908670446  cycles                   #    798.317 M/sec
    20816640224  instructions             #      0.996 IPC  
     7871360378  cache-references         #    300.538 M/sec
       11813272  cache-misses             #      0.451 M/sec

   26.370606238  seconds time elapsed

# Samples: 217594
#
# Overhead      Command                        Shared Object  Symbol
# ........  ...........  ...................................  ......
#
    34.25%      mplayer  ./mplayer                            [.] III_dequantize_sample
    17.89%      mplayer  ./mplayer                            [.] synth_1to1_MMX
    16.45%      mplayer  ./mplayer                            [.] dct36_3dnowex
    11.35%      mplayer  ./mplayer                            [.] dct64_MMX_3dnowex
     5.32%      mplayer  ./mplayer                            [.] T.293
     4.83%      mplayer  /lib/i686/cmov/libc-2.11.2.so        [.] memcpy
     0.65%      mplayer  [kernel]                             [k] __copy_to_user_ll
     0.59%      mplayer  ./mplayer                            [.] synth_1to1
\end{verbatim}

As a bonus, I compare also with good old mpg123 0.65 (the hacked i386 code). It is eerily close.
\begin{verbatim}
   28434.455457  task-clock-msecs         #      0.996 CPUs 
            298  context-switches         #      0.000 M/sec
              0  CPU-migrations           #      0.000 M/sec
            246  page-faults              #      0.000 M/sec
    22713949581  cycles                   #    798.818 M/sec
    27729289453  instructions             #      1.221 IPC  
     9204425341  cache-references         #    323.707 M/sec
        7354561  cache-misses             #      0.259 M/sec

   28.561796156  seconds time elapsed

# Samples: 234591
#
# Overhead      Command                  Shared Object  Symbol
# ........  ...........  .............................  ......
#
    40.85%       mpg123  ./src/mpg123                   [.] synth_1to1
    30.31%       mpg123  ./src/mpg123                   [.] III_dequantize_sample
    10.18%       mpg123  ./src/mpg123                   [.] dct64
     9.77%       mpg123  ./src/mpg123                   [.] dct36
     4.62%       mpg123  ./src/mpg123                   [.] do_layer3
     0.64%       mpg123  [kernel]                       [k] __copy_to_user_ll
     0.34%       mpg123  ./src/mpg123                   [.] dct12
     0.32%       mpg123  ./src/mpg123                   [.] III_get_scale_factors_1
     0.23%       mpg123  /lib/i686/cmov/libc-2.11.2.so  [.] __mempcpy
\end{verbatim}
I get just that bit of time for all that assembly code? SIMD and all?

For mpg123, the individual asm functions cannot be separated for some reason, so let's combine the numbers:
\begin{center}
\begin{tabular}{lrrrr}
         & time & time & time & time  \\
         & MPlayer & MPlayer & mpg123 & mpg123 \\
function & +mp3lib / s & +libmpg123 / s & trunk / s & 0.65-i386 / s
\\\hline
synth \& DCT            & 12.20 & 11.59 & 11.75 & 17.36 \\
III\_dequantize\_sample &  9.03 &  9.44 &  9.46 &  8.66 \\
T.293 / do\_layer3      &  1.40 &  1.36 &  1.43 &  1.31 \\
memcpy+to\_user         &  1.31 &  2.33 &  0.37 &  0.25 \\
sum of above            & 23.94 & 24.72 & 23.01 & 27.58 \\
diff to mpg123          &  0.93 & 1.71  &  0    &  4.57 \\
total runtime diff      &  1.79 & 2.89  &  0    &  3.98 
\end{tabular}
\end{center}
So, I can explain $59\%$ of the difference to mpg123 from these main contributors to computing time. And it is rather clear where that comes from: memcpy.
The same is even true for mp3lib, there, memcpy also plays a larger role.

A more subtle point is to be made about the assembly synths in mpg123 being a bit faster (yay, despite criticism about inefficient stack handling;-), while, again, MPlayer wins on the efficiency of GCC-generated code.
They managed to befriend to optimizer somehow.

Since I like to explore to the sidess before going for the bulls eye, I am going to have a look at what exactly is different in the sample dequantization.
This comparison is messy. Looking at that huge slab of code with CPU time distributed all over the place, but somewhat prominent on code portions that are rather identical between mp3lib and mpg123, I think it is not worth it no nail down the optimizer here. That half of a second can come and go.

Well, but what could be worth it is to investigate what the heck happened since mpg123 0.65 here. I am comparing to the plain C code for i386 (well, boosted a bit by the recent improvements / hack for GCC optimizer)!
I see the 3DNowExt assembly shaving off 5.61 seconds from the synth stuff, but here, 0.8 seconds are lost in the dequantization again.
Eh, I have to take a step back. I just glossed over half a second compared to mp3lib, I can ignore 0.8 seconds, too. Really, I have seen what GCC is capable of with the rather simple synth code.
I will not start a battle about 0.8 seconds in III\_dequantize\_sample. I need some lifetime left.
Let's be happy that in the sum, 14\% of decoding performance is gained from 3DNowExt.

What spooks me here, is that the plain 3DNow code makes mpg123 even slower than the beefed-up generic code:
\begin{verbatim}
   29208.504107  task-clock-msecs         #      0.995 CPUs 
            389  context-switches         #      0.000 M/sec
              0  CPU-migrations           #      0.000 M/sec
           4151  page-faults              #      0.000 M/sec
    23331202619  cycles                   #    798.781 M/sec
    26197929267  instructions             #      1.123 IPC  
     9571621109  cache-references         #    327.700 M/sec
       13034794  cache-misses             #      0.446 M/sec

   29.341045891  seconds time elapsed

# Samples: 248648
#
# Overhead      Command                  Shared Object  Symbol
# ........  ...........  .............................  ......
#
    31.26%    lt-mpg123  ./libs/libmpg123.so.0.33.1     [.] III_dequantize_sample
    30.25%    lt-mpg123  ./libs/libmpg123.so.0.33.1     [.] 0x00000000041684
    26.01%    lt-mpg123  .libs/libmpg123.so.0.33.1      [.] 0x00000000042611
     4.54%    lt-mpg123  .libs/libmpg123.so.0.33.1      [.] INT123_do_layer3
     0.89%    lt-mpg123  .libs/libmpg123.so.0.33.1      [.] INT123_synth_1to1_3dnow
     0.73%    lt-mpg123  /lib/i686/cmov/libc-2.11.2.so  [.] __mempcpy
     0.64%    lt-mpg123  [kernel]                       [k] __copy_to_user_ll
     0.64%    lt-mpg123  .libs/libmpg123.so.0.33.1      [.] III_get_scale_factors_1
\end{verbatim}
That makes 9.17 seconds for dequantization, 16.77 seconds for combined synth, that is a bit more and a bit less than the i386 C code.
All in all, one can say that any benefit from the assembly marginal, and, depending on the mood of the compiler, not worth anything. I feel that I could go down that road, a long way down, down into the abyss, but let's avoid that.
The fastest variant on this machine is what should be the fastest. Let's make MPlayer+libmpg123 live up to that.

\paragraph{I diagnosed some memcpy poisoning,}
so let's start with switching around I/O variants to avoid memcpy. I have something prepared there.
Some numbers for the differing variants (switches for packet-based input, plain demuxer reads, callback, feeder, callback with seek buffer).
The memcpy number is only the actual count for memcpy() now, not this kernel-to-userspace stuff (isn't that just that I am reading from /dev/shm?).
\begin{center}
\begin{tabular}{lrr}
I/O variant & packet (memcpy) & demux\_read (memcpy)
\\\hline
feeder   & 27.50 s (2.14 s) & 27.54 s (2.28 s) \\ 
callback & 27.35 s (1.98 s) & 27.58 s (2.02 s) \\
callback+seekbuffer
         & 27.79 s (2.31 s) & 27.52 s (2.33 s)
\end{tabular}
\end{center}
Big difference I see not, though some there is.
The most backwards-compatible and also parsingly flexible variant is second only to the plain callback variant that cannot peek ahead in streams.
I am willing to pay these 0.15 seconds for the stream parsing robustness and the fact that this API works with any release of libmpg123. Of course, the more I think of it, those 0.5\% aren't {\em nothing}. Sheesh.
Oh, and don't forget that the numbers have some leeway. An earlier measurement the packet feeder took 27.47 seconds.
Well, rounding to 1/10th seconds would be reasonable, when jumping to final conclusions. As I am groping for any hint here, it should be fine to also look one digit further for the wobbly fine print.

Focus on the bigger picture: There is more to it than the input buffering with libmpg123.
Or, hm... have a look at the call graph for the packet feeder, compared to plain mpg123 and mp3lib.

So, the rare memcpy calls in plain mpg123 are rooted in the WAV writing, not decoding.
With mp3lib, the call graph doesn't give sensible information as to where the calls orginate. 99.60\% from (nil). Great.
What is always apparent, is the presence of those fast\_memcpy() calls. They amount to 0.46\% instead compared to 4.84\% for the libc one.
They don't occur in the call graph for libmpg123 usage from MPlayer.
There, at least, is a sensible trace of the plain memcpy() calls: 
\begin{verbatim}
     8.02%  mplayer-feeder-  /lib/i686/cmov/libc-2.11.2.so                                        [.] memcpy
                |          
                |--58.67%-- 0x10000
                |          
                |--41.15%-- decode_a_bit
                 --0.18%-- [...]
\end{verbatim}
At least half of them can be accounted for. Not surprisingly, that's the feeding of data via mpg123\_decode().
Half of the good two seconds spent in memcpy would already suffice to get on par with mp3lib...

But, since there is not one second difference from the feeder to the plain callback ... I guess demux\_read\_data() also employs simple memcpy for the same task. Hm, so...
Wait! I wanted to look into that anyway, and I see confirmation in the call graph for the callback variant: There, it's mpg123\_read() and decode\_audio().
Yes, we come to the point where I will ponder variants to get the audio data {\em out} of libmpg123. 

Well, there is mpg123\_replace\_reader() that should help to prevent copies, but after hacking that in, I reduced the memcpy usage, all right, but the total time didn't go down. Compare the callback variant with packets: 27.65 seconds runtime, of that 1.24 seconds memcpy().
This should be half a second faster, but it's third of a second slower!

I get that down to 27.52 s by lying  about buffer sizes... that really should match what mp3lib does (directly decoding to a 4608-byte caller-provided buffer), but it doesn't.
I got 9.49 s for the dequantization, 12.32 s for the synth stuff, 1.41 for do\_layer and 1.24 s for memcpy.
So there is no net gain at all, only more messy code.

This very much looks like a dead end here. Let's do a comparison with linking to libmpg123 statically. Interesting, that gets it down to 26.80 s. That's about 0.4 seconds more than mp3lib.

Switching back to plain normal feeder mode with packets produces 27.65 s. Activating framewise work again... that's 27.03 s. The whole deal, callback, packets, seekbuffer, framewise, is at 27.01 s.
Am I being ridiculous? Well, it's close.

Well, working around memcpy is not the total breakthrough... I still have the option to use a faster memcpy in libmpg123. What about copying as 32 bit integers for a simple start? Nah, that naive approach does not improve things. So glibc memcpy is not {\em that} dumb.

Closest I can reach is 26.69 s via callback and framewise, without packets, seekbuffer and mean bitrate calculation.

\paragraph{On the K6-III+}
I finally an claim victory for framewise decoding with a static libmpg123.
It's indeed a bit faster than mp3lib. Not so on the Duron (also with \verb:-march=athlon:), there, libmpg123 can only come close, but never overtake.
But this is with this certain setup / tool chain. Who knows what gcc 4.6 would do?

Native binaries on Duron and K6-III+ ... and my Laptop Core2Duo at 800 MHz, 64 bit mode (on the K6 it's user time measured by the \verb:time: command, the others via perf).
\begin{tabular}{lrrr}
build                     & Duron   &  K6-III+ & Core 2 \\
\hline
variant1-dynamic (1.12.1) & 27.74 s &  57.46 s &         \\
variant1-dynamic (1.13.4) &         &          & 16.72 s \\
variant1-dynamic (trunk)  & 27.81 s &  57.05 s & 16.87 s \\
variant2-dynamic          & 27.10 s &  55.92 s & 16.54 s \\
variant2-static           & 27.01 s &  53.92 s & 16.51 s \\
mp3lib                    & 26.45 s &  56.47 s & 18.56 s \\
ffmp3                     & 73.52 s & 115.08 s & 22.00 s \\
ffmp3float                & 30.74 s & 118.19 s & 12.26 s \\
variant2-float            &         &          & 18.76 s
\end{tabular}
So, the comparisong between mp3lib and libmpg123 does look very good on the K6-III+, not so good on the Duron.
Would be interesting to have the comparison with Ivan's Athlon.
On the Core2Duo there is not discussion because of the better SSE code anyway. Well, apart from the strange numbers for ffmp3float, but I'm getting to that later.

Here, the net result is that I can be 4.5\% faster on the K6-III+ and down to 2.1\% slower on the Duron (from 5.1\%). Ivan had 8\% loss (user time, 53.4 s agains 49.4 s).
Oh, and the final comparison on the K6-III+ using mplayer's own benchmark mechanism:
\begin{verbatim}
altwarentester@k6-3:~/mplayer$ for i in mpg123 mp3; do ./mplayer-variant2-static -loop 5 \
  -ao pcm:fast:file=/dev/null -benchmark -quiet -ac mp3 \
  /dev/shm/convergence_-_points_of_view/10\ -\ Train\ to\ leave.mp3 \
  2>&1 | grep BENCH; echo "$i ^^^^^"; done
BENCHMARKs: VC:   0.000s VO:   0.000s A:   6.653s Sys:   0.051s =    6.704s
BENCHMARK%: VC:  0.0000% VO:  0.0000% A: 99.2399% Sys:  0.7601% = 100.0000%
BENCHMARKs: VC:   0.000s VO:   0.000s A:   6.659s Sys:   0.053s =    6.712s
BENCHMARK%: VC:  0.0000% VO:  0.0000% A: 99.2089% Sys:  0.7911% = 100.0000%
BENCHMARKs: VC:   0.000s VO:   0.000s A:   6.657s Sys:   0.052s =    6.708s
BENCHMARK%: VC:  0.0000% VO:  0.0000% A: 99.2291% Sys:  0.7709% = 100.0000%
BENCHMARKs: VC:   0.000s VO:   0.000s A:   6.685s Sys:   0.052s =    6.737s
BENCHMARK%: VC:  0.0000% VO:  0.0000% A: 99.2267% Sys:  0.7733% = 100.0000%
BENCHMARKs: VC:   0.000s VO:   0.000s A:   6.653s Sys:   0.050s =    6.703s
BENCHMARK%: VC:  0.0000% VO:  0.0000% A: 99.2524% Sys:  0.7476% = 100.0000%
mpg123 ^^^^^
BENCHMARKs: VC:   0.000s VO:   0.000s A:   6.600s Sys:   0.055s =    6.655s
BENCHMARK%: VC:  0.0000% VO:  0.0000% A: 99.1722% Sys:  0.8278% = 100.0000%
BENCHMARKs: VC:   0.000s VO:   0.000s A:   6.592s Sys:   0.048s =    6.640s
BENCHMARK%: VC:  0.0000% VO:  0.0000% A: 99.2755% Sys:  0.7245% = 100.0000%
BENCHMARKs: VC:   0.000s VO:   0.000s A:   6.600s Sys:   0.051s =    6.650s
BENCHMARK%: VC:  0.0000% VO:  0.0000% A: 99.2378% Sys:  0.7622% = 100.0000%
BENCHMARKs: VC:   0.000s VO:   0.000s A:   6.604s Sys:   0.052s =    6.656s
BENCHMARK%: VC:  0.0000% VO:  0.0000% A: 99.2248% Sys:  0.7752% = 100.0000%
BENCHMARKs: VC:   0.000s VO:   0.000s A:   6.596s Sys:   0.050s =    6.646s
BENCHMARK%: VC:  0.0000% VO:  0.0000% A: 99.2493% Sys:  0.7507% = 100.0000%
mp3 ^^^^^
\end{verbatim}
Damn, that's a mean user time of 6.66 s against 6.59 s for mp3lib. How subtle is that?
What is the equivalent timing of the same track 10 times using my method?
\begin{verbatim}
altwarentester@k6-3:~/mplayer$ for i in mpg123 mp3
  do time ./mplayer-variant2-static -loop 5 -ao pcm:fast:file=/dev/null \
  -quiet -ac mp3 \
  /dev/shm/convergence_-_points_of_view/10\ -\ Train\ to\ leave.mp3 \
  2>&1 > /dev/null; echo "$i ^^^^^"; done

real    0m34.197s
user    0m33.498s
sys     0m0.636s
mpg123 ^^^^^

real    0m34.145s
user    0m33.446s
sys     0m0.656s
mp3 ^^^^^
\end{verbatim}
Oh, funny. See what I did there? One should actually switch decoders. But still, interesting to see that I was about to diagnose a performance difference of 1\% between mp3lib and itself.
This is the proper measurement:
\begin{verbatim}
altwarentester@k6-3:~/mplayer$ for i in mpg123 mp3; do ./mplayer-variant2-static\
  -loop 5 -ao pcm:fast:file=/dev/null -benchmark -quiet -ac $i \
  /dev/shm/convergence_-_points_of_view/10\ -\ Train\ to\ leave.mp3 \
  2>&1 | grep BENCH; echo "$i ^^^^^"; done
BENCHMARKs: VC:   0.000s VO:   0.000s A:   6.393s Sys:   0.067s =    6.460s
BENCHMARK%: VC:  0.0000% VO:  0.0000% A: 98.9636% Sys:  1.0364% = 100.0000%
BENCHMARKs: VC:   0.000s VO:   0.000s A:   6.314s Sys:   0.061s =    6.375s
BENCHMARK%: VC:  0.0000% VO:  0.0000% A: 99.0432% Sys:  0.9568% = 100.0000%
BENCHMARKs: VC:   0.000s VO:   0.000s A:   6.327s Sys:   0.067s =    6.394s
BENCHMARK%: VC:  0.0000% VO:  0.0000% A: 98.9583% Sys:  1.0417% = 100.0000%
BENCHMARKs: VC:   0.000s VO:   0.000s A:   6.316s Sys:   0.060s =    6.376s
BENCHMARK%: VC:  0.0000% VO:  0.0000% A: 99.0572% Sys:  0.9428% = 100.0000%
BENCHMARKs: VC:   0.000s VO:   0.000s A:   6.323s Sys:   0.064s =    6.387s
BENCHMARK%: VC:  0.0000% VO:  0.0000% A: 99.0008% Sys:  0.9992% = 100.0000%
mpg123 ^^^^^
BENCHMARKs: VC:   0.000s VO:   0.000s A:   6.682s Sys:   0.057s =    6.739s
BENCHMARK%: VC:  0.0000% VO:  0.0000% A: 99.1586% Sys:  0.8414% = 100.0000%
BENCHMARKs: VC:   0.000s VO:   0.000s A:   6.693s Sys:   0.057s =    6.750s
BENCHMARK%: VC:  0.0000% VO:  0.0000% A: 99.1543% Sys:  0.8457% = 100.0000%
BENCHMARKs: VC:   0.000s VO:   0.000s A:   6.671s Sys:   0.055s =    6.727s
BENCHMARK%: VC:  0.0000% VO:  0.0000% A: 99.1761% Sys:  0.8239% = 100.0000%
BENCHMARKs: VC:   0.000s VO:   0.000s A:   6.671s Sys:   0.058s =    6.729s
BENCHMARK%: VC:  0.0000% VO:  0.0000% A: 99.1383% Sys:  0.8617% = 100.0000%
BENCHMARKs: VC:   0.000s VO:   0.000s A:   6.670s Sys:   0.055s =    6.725s
BENCHMARK%: VC:  0.0000% VO:  0.0000% A: 99.1802% Sys:  0.8198% = 100.0000%
mp3 ^^^^^
\end{verbatim}
That's a mean of 6.33 s for libmpg123 against 6.68 s for mp3lib, a decrease in decoding time of 5.2\%.
Of course, we got quite some variation here and I'm too lazy to compute the standard deviation, but it is apparent that mpg123 has a consistent lead here.

\paragraph{What is the deal with ffmp3float on my Core2Duo?}
I'm comparing with libmpg123 in floating point mode (also assembly optimized).
This does not seem right. The waves look good.
How can this be suddenly that fast? What kind of 64 bit assembly magic is part of that?

It is suspicious that ffmp3float is not consistent. Repeated runs show differing numbers. Most interestingly such a run:
\begin{verbatim}
      10817,626571 task-clock                #    0,049 CPUs utilized          
            16.540 context-switches          #    0,002 M/sec                  
                19 CPU-migrations            #    0,000 M/sec                  
             3.377 page-faults               #    0,000 M/sec                  
     8.544.770.425 cycles                    #    0,790 GHz                    
   <not supported> stalled-cycles-frontend 
   <not supported> stalled-cycles-backend  
     8.038.105.826 instructions              #    0,94  insns per cycle        
     1.302.988.705 branches                  #  120,451 M/sec                  
        41.415.237 branch-misses             #    3,18% of all branches        

     222,249003389 seconds time elapsed
\end{verbatim}
Well, how do you come from 12 seconds to 222? Easy, you actually play in realtime.
\begin{verbatim}
MPlayer SVN-r34365-4.6.1 (C) 2000-2011 MPlayer Team

Playing pcm:file=/dev/null.
File not found: 'pcm:file=/dev/null'
Failed to open pcm:file=/dev/null.


Playing /dev/shm/convergence_-_points_of_view/03 - Listen.mp3.
\end{verbatim}
This looks rather corrupt from the beginning. The command line is always the same, the interpretation differs.

Valgrind is no help with my MPlayer build:
\begin{verbatim}
valgrind: m_debuginfo/readdwarf.c:2162 (copy_convert_CfiExpr_tree):
  Assertion 'srcix >= 0 && srcix < VG_(sizeXA)(srcxa)' failed.
\end{verbatim}
Well, I'm not here for debugging ffmp3float. Perhaps that's a bug fixed since I did my checkout of MPlayer / ffmpeg.

\section{Summary}

Let me summarize the results. Where did I start, where did I end up? 
\begin{center}
\begin{tabular}{lrrr}
run & user time K6-III+ / s & user time Duron / s & ratio K6 / Duron \\
run & new (old) & new (old) / s & new (old) \\
\hline
mplayer-mp3lib & 56.5 (56.8) & 26.5 (25.4) & 2.13 (2.24) \\
mplayer-mpg123 & 53.9 (67.0) & 27.0 (28.7) & 2.00 (2.33)
\end{tabular}
\end{center}
Eh... how did I arrive at 67 seconds with libmpg123 before? Was that a typo? Or did I actually improve \verb:ad_mpg123: that much, even with old libmpg123? I measured 57.5 seconds now. Did I mean 57 instead of 67? Damn.

I did omit the \verb:-g: flag in the latest binaries, to get the best result. Did it make that difference? Checking again with old \verb:ad_mpg123: and the flags without \verb:-g:. Gah! It needs 56.2 seconds.
I have the massive suspicion that I had a typo in the beginning.
But everything's possible. Thus, another build of MPlayer, now with the exact same flags as in the beginning.

\end{document}

